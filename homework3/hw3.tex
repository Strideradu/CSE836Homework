%=======================02-713 LaTeX template, following the 15-210 template==================
%
% You don't need to use LaTeX or this template, but you must turn your homework in as
% a typeset PDF somehow.
%
% How to use:
%    1. Update your information in section "A" below
%    2. Write your answers in section "B" below. Precede answers for all 
%       parts of a question with the command "\question{n}{desc}" where n is
%       the question number and "desc" is a short, one-line description of 
%       the problem. There is no need to restate the problem.
%    3. If a question has multiple parts, precede the answer to part x with the
%       command "\part{x}".
%    4. If a problem asks you to design an algorithm, use the commands
%       \algorithm, \correctness, \runtime to precede your discussion of the 
%       description of the algorithm, its correctness, and its running time, respectively.
%    5. You can include graphics by using the command \includegraphics{FILENAME}
%
\documentclass[11pt]{article}
\usepackage{amsmath,amssymb,amsthm}
\usepackage{graphicx}
\usepackage[margin=1in]{geometry}
\usepackage{fancyhdr}
\setlength{\parindent}{0pt}
\setlength{\parskip}{5pt plus 1pt}
\setlength{\headheight}{13.6pt}
\usepackage{algorithm}
\usepackage{algpseudocode}
\usepackage{float}

\newcommand*{\permcomb}[4][0mu]{{{}^{#3}\mkern#1#2_{#4}}}
\newcommand*{\perm}[1][-3mu]{\permcomb[#1]{P}}
\newcommand*{\comb}[1][-1mu]{\permcomb[#1]{C}}

\newcommand\question[2]{\vspace{.25in}\hrule\textbf{#1: #2}\vspace{.5em}\hrule\vspace{.10in}}
\renewcommand\part[1]{\vspace{.10in}\textbf{(#1)}}
\newcommand\algorithmcode{\vspace{.10in}\textbf{Algorithm: }}
\newcommand\correctness{\vspace{.10in}\textbf{Correctness: }}
\newcommand\runtime{\vspace{.10in}\textbf{Running time: }}
\pagestyle{fancyplain}
\lhead{\textbf{\NAME\ (\ANDREWID)}}
\chead{\textbf{HW\HWNUM}}
\rhead{CSE 836, \today}

\begin{document}\raggedright
%Section A==============Change the values below to match your information==================
\newcommand\NAME{Nan Du}  % your name
\newcommand\ANDREWID{dunan}     % your andrew id
\newcommand\HWNUM{3}              % the homework number
%Section B==============Put your answers to the questions below here=======================

% no need to restate the problem --- the graders know which problem is which,
% but replacing "The First Problem" with a short phrase will help you remember
% which problem this is when you read over your homeworks to study.

\question{1}{Pseudo Code for Algorithm} 

\part{a} 


\begin{algorithm}[H]
	\caption{Build Suffix Array}\label{suffixarray}
	\begin{algorithmic}[1]
		\Function{SA}{$s$}\Comment{String s as input}
		\State satups$ \gets $ [empty array] 
		\For{i in $ |s| $}
			\State Add s[i:$ |s| $] to satups
		\EndFor
		\State sort the satups based on lexicographically order
		\State SA[sort index] $ gets $ original suffix index
		\State\Return SA
		\EndFunction

	\end{algorithmic}
\end{algorithm}


\part{b}

\begin{algorithm}[H]
	\caption{Convert Suffix Array To BWT}\label{bwt}
	\begin{algorithmic}[1]
		\Function{BWT}{$sa, text$}\Comment{Suffix array sa and the text as input}
		\State bwt$ \gets $ [empty array] 
		\For{i in $ |sa| $}
			\State Add text[sa[i] - 1] to bwt
		\EndFor
		\State\Return bwt
		\EndFunction

	\end{algorithmic}
\end{algorithm}

\part{c}

\begin{algorithm}[H]
	\caption{Construct FM index's OCC}\label{fmocc}
	\begin{algorithmic}[1]
		\Function{rank}{$bwt$}\Comment{BWT bwt as input}
		\State tots $ \gets $ new dict
		\State ranks $ \gets $ new array
		\For{c in bwt}
		\If{c not in tots}
		\State let tots[c]$ \gets $0
		\EndIf
		\State Add tots[c] to ranks
		\State tots[c] $ \gets $tots[c]+1
		\EndFor
		\State\Return ranks, tots
		\EndFunction
		
	\end{algorithmic}
\end{algorithm}

\begin{algorithm}[H]
	\caption{Construct FM index's C}\label{fmc}
	\begin{algorithmic}[1]
		\Function{count}{$tots$}\Comment{tots as input}
		\State totc $ \gets $ 0
		\State first $ \gets $ new array
		\State sort tots with its key
		\For{c, count in tots}

		\State first[c]$ \gets $ totc
		\State totc $ \gets $ totc + count


		\EndFor
		\State\Return first
		\EndFunction
		
	\end{algorithmic}
\end{algorithm}
	
\part{d}

\begin{algorithm}[H]
	\caption{Update search range}\label{update}
	\begin{algorithmic}[1]
		\Function{Update}{$pi, start, end$}\Comment{Character pi, row index start, and row index end as input}
		\State start$ \gets $ ranks(pi, start - 1) + first(pi)

		\State end$ \gets $ ranks(pi, end) + first(pi) - 1

		\State\Return start, end
		\EndFunction
		
	\end{algorithmic}
\end{algorithm}

\begin{algorithm}[H]
	\caption{Check if reach the end}\label{check}
	\begin{algorithmic}[1]
		\Function{Check}{$start, end$}\Comment{ow index start, and row index end as input}
		\State range$ \gets $ suffixarray[start: end+1]
		
		\For{index in range}
		\If {text[index - 1] $ = $ \$ } 
		\State \Return index
		\EndIf
		\EndFor
		\EndFunction
		
	\end{algorithmic}
\end{algorithm}

\begin{algorithm}[H]
	\caption{Find overlap between pattern's prefix and index' suffix}\label{overlap}
	\begin{algorithmic}[1]
		\Function{Update}{$p, fmindex, threshold$}\Comment{Pattern p, FM index fmindex, and overlap threshold as input}
		\State reverse input pattern p
		\State start$ \gets $ 0
		
		\State end$ \gets $ length of indexed text -1
		\State overlapsize$ \gets $ 0
		
		\For{c in reversed p}
		\State start, end $ \gets $ Update(c, start, end)
		\State overlapsize$ \gets $ overlapsize++
		\If {overlapsize > threshold}
		\State Check(start, end)
		\If {Check return value} stop and return the overlap size
		\EndIf
		\EndIf
		\EndFor
		\EndFunction
		
	\end{algorithmic}
\end{algorithm}

\question{2}{Running Time}

On HPCC, it took about 40 seconds using 1 thread to finish the all against all compare on the given HIV reads dataset

\question{3}{Memory Usage}
For the all against all compare on the given HIV reads dataset, my program use 131 MB memory


\end{document}
